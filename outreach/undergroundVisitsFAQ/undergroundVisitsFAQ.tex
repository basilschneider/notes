\documentclass{atlasnote}

\usepackage{atlasphysics}
\usepackage{subfigure}
\usepackage{authblk}
\usepackage{url}

\renewcommand\Authands{, } % avoid ``. and'' for last author
\renewcommand\Affilfont{\itshape\small} % affiliation formatting

\skipbeforetitle{-100pt}

\title{FAQ for ATLAS guides}

\author[1]{B.~Schneider} 

\affil[1]{TRIUMF}

\abstracttext{This FAQ provides a list of typical questions asked by the public during ATLAS visits. It focuses on numerical answers that are not important for
our everyday work but interesting for the layman. Comparisons to everyday life are provided. Rigorousness is not required.}

\newenvironment{question} % no indenting, bold and new paragraph
{\noindent\bfseries}
{\par}

\newenvironment{answer} % v spacing at the end
{}
{\vspace*{10pt}}

\newenvironment{faq} % environment to prevent page breaks between question and answer (wrap around \begin{question} and \end{answer})
{\par\noindent\begin{minipage}{\linewidth}}
    {\end{minipage}\par}

\begin{document}

\section{CERN/LHC}

\begin{faq}
    \begin{question}
        How long lasts the hydrogen bottle used for the colliding protons?
    \end{question}
    \begin{answer}
        To empty the hydrogen bottle with its $140$~g of hydrogen, it would take about $150$~million years. Nonetheless, to ensure stable pressure, it is
        replaced twice a year.
    \end{answer}
\end{faq}

\begin{faq}
    \begin{question}
        How many work hours were spent to built the LHC?
    \end{question}
    \begin{answer}
        I don't know.
    \end{answer}
\end{faq}

\begin{faq}
    \begin{question}
        At what proton energies do the different accelerator systems operate?
    \end{question}
    \begin{answer}
        Linac2 accelerates the protons to $50$~MeV, the PS booster to $1.4$~GeV, the PS to $25$~GeV and the SPS to $450$~GeV. 
    \end{answer}
\end{faq}

\begin{faq}
    \begin{question}
        What's the timeline of the LHC?
    \end{question}
    \begin{answer}
        Early 1980's while LEP was built, first plans about a possible successor circulated. In December 1994 the LHC was approved by the CERN
        council. Building started shortly therafter. First collisions happened at 23$^{rd}$~November 2009. The four large experiments were approved between 1996
        and 1998. 
    \end{answer}
\end{faq}

\begin{faq}
    \begin{question}
        How long does it take to get a stable beam?
    \end{question}
    \begin{answer}
        It takes 4'20" to fill one LHC ring, the acceleration of the protons to $7$~GeV takes $20$~minutes, the stable beam is used for collisions for several
        hours before it is dumped. Before a proton is dumped, it has travelled the distance to Neptun and back. After a beam is dumped, it takes roughly 2 hours
        to get stable beams again.
    \end{answer}
\end{faq}

\begin{faq}
    \begin{question}
        How deep under ground is the LHC, ATLAS?
    \end{question}
    \begin{answer}
        The mean depth is $\sim 100$~m, the depth varies between $175$~m (under the Jura) and $50$~m (towards Lake Geneva). ATLAS is $92$~m under ground.
    \end{answer}
\end{faq}

\begin{faq}
    \begin{question}
        What about LEP?
    \end{question}
    \begin{answer}
        LEP operated from 1989 to 2000. It's still the most powerful lepton collider ever built. Highest energy achieved was $209$~GeV.
    \end{answer}
\end{faq}

\begin{faq}
    \begin{question}
        It is often said, that the LHC probes the early phase of the Universe. But at what time after the big bang was the typical energy density comparable to the
        energies reached at the LHC?
    \end{question}
    \begin{answer}
        $10^{-12}$~s
    \end{answer}
\end{faq}

\begin{faq}
    \begin{question}
        How much did the LHC cost?
    \end{question}
    \begin{answer}
        The construction costs of the LHC~-~including machine R \& D and injectors, tests and pre-operation~-~was about $5$~billion CHF. About a quarter was spent on
        personnels, the rest for materials. Additionally CERN shares some of the expenses of the detectors and computing facilities. In total $6.51$~billion CHF was
        spent.
    \end{answer}
\end{faq}

\begin{faq}
    \begin{question}
        How much money does CERN contribute to the material budget of the different experiments?
    \end{question}
    \begin{answer}
        The experiments are individual entities, but CERN is a member of each experiment. It contributes $20$~\% to the material budget of CMS and LHCb, $16$~\%
        for ALICE, $14$~\% for ATLAS and $30$~\% for TOTEM.
    \end{answer}
\end{faq}

\begin{faq}
    \begin{question}
        What's the length of the tunnel?
    \end{question}
    \begin{answer}
        $26.659$~km
    \end{answer}
\end{faq}

\begin{faq}
    \begin{question}
        What are collision energies for ion runs?
    \end{question}
    \begin{answer}
        $2.76$~TeV per ion, this gives a total of $1150$~TeV.
    \end{answer}
\end{faq}

\begin{faq}
    \begin{question}
        What's the temperature in a collision?
    \end{question}
    \begin{answer}
        I don't know.
    \end{answer}
\end{faq}

\begin{faq}
    \begin{question}
        What's the temperature of the superconducting electromagnets?
    \end{question}
    \begin{answer}
        The LHC magnets use superfluid helium at $1.9$~K. This is even less than the temperature at outer space ($2.7$~K).
    \end{answer}
\end{faq}

\begin{faq}
    \begin{question}
        How many magnets are used in the LHC?
    \end{question}
    \begin{answer}
        The LHC has a large variety of different magnets, in total about $9593$~are used. The biggest ones are the $1232$~dipoles. Each dipole is $15$~m long and
        weighs around $35$~t.
    \end{answer}
\end{faq}

\begin{faq}
    \begin{question}
        How many protons are circulating in the LHC at the same time?
    \end{question}
    \begin{answer}
        A complete fill has $2808$~proton bunches, each having $1.1 \cdot 10^{11}$~protons.
    \end{answer}
\end{faq}

\begin{faq}
    \begin{question}
        How strong is the magnetic field of the superconducting dipoles?
    \end{question}
    \begin{answer}
        The peak magnetic dipole field is at $8.33$~T.
    \end{answer}
\end{faq}

\begin{faq}
    \begin{question}
        What is the distance between bunches?
    \end{question}
    \begin{answer}
        The minimal distance is $25$~ns or about $7$~m.
    \end{answer}
\end{faq}

\begin{faq}
    \begin{question}
        What is the luminosity?
    \end{question}
    \begin{answer}
        The design luminosity is $10^{34}$~cm$^{-2}$s$^{-1}$, the peak luminosity reached is ???
    \end{answer}
\end{faq}

\begin{faq}
    \begin{question}
        How many turns and collisions happen per second?
    \end{question}
    \begin{answer}
        The revolution frequency of a proton bunch is $11.245$~kHz. The number of collisions per second obviously depends on bunch crossing and luminosity
        (pile-up), at designated values, there would be $600$~million collisions per second. 
    \end{answer}
\end{faq}

\begin{faq}
    \begin{question}
        Do Tides influence the LHC?
    \end{question}
    \begin{answer}
        Yes. By new moon and full moon, the crust in the Geneva region rises by some $25$~cm. This movement causes a variation of $1$~mm in the circumference of
        the LHC and this produces changes in beam energy. The effect is not crucial, though, since the collision energy of the partons is not known (as opposed to
        the situation at LEP).
    \end{answer}
\end{faq}

\begin{faq}
    \begin{question}
        What are the dimensions of a bunch?
    \end{question}
    \begin{answer}
        Usually, it's a few cm's long and about $1$~mm wide, at collision point it is squeezed to $16 \, \mu$m to allow for a greater chance of collision.
    \end{answer}
\end{faq}

\begin{faq}
    \begin{question}
        How much data is produced by the LHC?
    \end{question}
    \begin{answer}
        From the $600$~million events per second only a few $100$'s~are stored. Thus, more than $99.9999$~\% of the collision data is thrown away
        immediately. The recorded
        events generate about $700$~MB/s or $15$~PB/year of data. If all data would be recorded, it would be more than $3$~PB/s or about $7.5$~million PB/year
        ($=7.5$~zettabyte/year).
    \end{answer}
\end{faq}

\begin{faq}
    \begin{question}
        What is the LHC power consumption?
    \end{question}
    \begin{answer}
        It is around $120$~MW ($230$~MW for all CERN), which corresponds more or less to the power consumption for households in the Canton of Geneva. The
        estimated yearly energy consumption is about $8 \cdot 10^5$~MWh, including site based load and experiments. The total yearly cost is therefore about $25$
        million CHF.
    \end{answer}
\end{faq}

\begin{faq}
    \begin{question}
        How large is the radiation generated at CERN?
    \end{question}
    \begin{answer}
        The radioactivity released by the LHC is less than $10 \, \mu$Sv/year. As a comparison, the natural radioactivity is about $2400 \, \mu$Sv/year in
        Switzerland, a flight from Europe to Los Angeles accounts for about $100 \, \mu$Sv. Despite its low levels, the LHC's radioactivity is closely monitored
        by CERN, which conducts a rigorous control programme that includes over 200 monitoring stations, in agreement with the Swiss and French authorities.
    \end{answer}
\end{faq}

\begin{faq}
    \begin{question}
        What's the amount of helium needed to cool down the LHC?
    \end{question}
    \begin{answer}
        Around $120$~t.
    \end{answer}
\end{faq}

\begin{faq}
    \begin{question}
        What is the total energy of the beam?
    \end{question}
    \begin{answer}
        The total energy in each beam at maximum energy is about $350$~MJ, which is about as energetic as the TGV travelling at $150$~km/h. This is enough
        energy to melt around $500$~kg of copper. The total energy stored in the LHC magnets is about $11$~GJ.
    \end{answer}
\end{faq}

\begin{faq}
    \begin{question}
        How much filament is used in the LHC magnet coils?
    \end{question}
    \begin{answer}
        If all the filaments in the magnet coils were unravelled, they would stretch to the Sun and back five times with enough left over for $150$~trips to the
        Moon and back.
    \end{answer}
\end{faq}

\begin{faq}
    \begin{question}
        When was the tunnel excavated?
    \end{question}
    \begin{answer}
        It was excavated at 1988. The two ends met up to within $1$~cm.
    \end{answer}
\end{faq}

\begin{faq}
    \begin{question}
        What's the pressure inside the beam pipe?
    \end{question}
    \begin{answer}
        $10^{-13}$~bar, about ten times lower than on the Moon.
    \end{answer}
\end{faq}

\begin{faq}
    \begin{question}
        How many member states does CERN have?
    \end{question}
    \begin{answer}
        $21$, the last to join being Israel in 2013. CERN has also associate members and states with observer status.
    \end{answer}
\end{faq}

\begin{faq}
    \begin{question}
        What computing power is used at CERN?
    \end{question}
    \begin{answer}
        The CERN computer centre consists of $25000$~computers. The grid adds more than $100000$~computers. The grid runs more than $1$~million jobs per day.
    \end{answer}
\end{faq}

\begin{faq}
    \begin{question}
        How many people are working at CERN?
    \end{question}
    \begin{answer}
        CERN employs just over $2400$~people. Some $10000$~scientists from over $113$~countries come to CERN for their research.
    \end{answer}
\end{faq}

\begin{faq}
    \begin{question}
        When was CERN established?
    \end{question}
    \begin{answer}
        29\textsuperscript{th} September 1954.
    \end{answer}
\end{faq}

\begin{faq}
    \begin{question}
        What are the biggest scientific achievements at CERN?
    \end{question}
    \begin{answer}
        \begin{tabular}{l l}
            1973: & The discovery of neutral currents in the Gargamelle bubble chamber\\
            1983: & The discovery of W and Z bosons in the UA1 and UA2 experiments\\
            1989: & The determination of the number of light neutrino families at the Large Electron-Positron\\
            & Collider (LEP) operating on the Z boson peak\\
            1989: & The World Wide Web began as a CERN project called ENQUIRE, initiated by\\
            & Tim Berners-Lee and Robert Cailliau\\
            1995: & The first creation of antihydrogen atoms in the PS210 experiment\\
            1999: & The discovery of direct CP violation in the NA48 experiment\\
            2010: & The isolation of 38 atoms of antihydrogen\\
            2011: & Maintaining antihydrogen for over 15 minutes\\
            2012: & A boson with mass around 125 GeV consistent with long-sought Higgs boson\\
        \end{tabular}
    \end{answer}
\end{faq}

\section{ATLAS}

\begin{faq}
    \begin{question}
        How many people work at ATLAS?
    \end{question}
    \begin{answer}
        The ATLAS collaboration consists of about $3000$~scientists, working in $177$~institutions from $38$~countries.
    \end{answer}
\end{faq}

\begin{faq}
    \begin{question}
        What are the key figures of ATLAS?
    \end{question}
    \begin{answer}
        ATLAS is $46$~m long, $25$~m high and $25$~m wide. It weighs $7000$~tons and its material cost was $540$~million CHF. ATLAS is the biggest of the
        CERN detectors (CMS the heaviest, with about $12000$~tons).
    \end{answer}
\end{faq}

\begin{faq}
    \begin{question}
        How many cables are used in ATLAS?
    \end{question}
    \begin{answer}
        ATLAS uses about $100$~million electronic channels and about $3000$~km of cables.
    \end{answer}
\end{faq}

\begin{faq}
    \begin{question}
        What are the key figures of the superconducting magnet coils?
    \end{question}
    \begin{answer}
        They are $25$~m long and weigh over $100$~t. They work at a temperature of $4$~K, with a magnetic field of $4$~T, storing more than $1$~GJ of energy.
    \end{answer}
\end{faq}

\begin{faq}
    \begin{question}
        What are the figures of the surface hall (SX1)?
    \end{question}
    \begin{answer}
        SX1 is $84$~m long, $24$~m wide, $18$~m high. Two cranes, each can carry $140$~t, can lower down pieces through the two access shafts. The heavies
        piece to lower down were the liquid Argon endcaps ($270$~t each).
    \end{answer}
\end{faq}






\begin{thebibliography}{1}
    \bibitem{atlas-factsheet} ATLAS Fact sheet \url{http://www.atlas.ch/fact_sheets.html}
    \bibitem{lhc-milestones} LHC milestones \url{http://lhc-milestones.web.cern.ch/LHC-Milestones/}
    \bibitem{dest-universe} Destination Universe: The Incredible Journey of a Proton in the Large Hadron Collider \url{http://cds.cern.ch/record/1259890}
    \bibitem{lhc-guide} LHC: The guide \url{http://cds.cern.ch/record/1165534/}
\end{thebibliography}

\end{document}
